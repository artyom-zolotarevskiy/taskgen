\documentclass{article}
\usepackage{pgf,pgffor}% http://ctan.org/pkg/{pgf,pgffor}
\usepackage{xfp}
\usepackage{array}
\input{random}
\usepackage{lcg,calc}
\usepackage[russian]{babel}
\usepackage{amsfonts}
\usepackage{ifthen}
\usepackage[depythontex]{pythontex}
%\restartpythontexsession{\thesection}
%\usepackage{verbatim}   % for the comment environment
%%%%%%%%%%%%%%%%%%%%%%%%%%%%%%%%%%%
\newenvironment{problem}[1][]{\rmfamily}{}{%
  \small
}{%
}

\newcounter{answercounter}

% if you want to see the solution
%\newcommand{\hidesolution}{show them}
\newcommand{\setplaceholders}{}
%%%%%%%%%%%%%%%%%%%%%%%%%%%%%%%%%%%%%%%
\newenvironment{solution}%
{%  
  \setcounter{answercounter}{1}
}%
{%
}
%%%%%%%%%%%%%%%%%%%%%%%%%%%%%%%%%
\setlength{\parindent}{0pt}
\usepackage[in]{fullpage}

% параметры:
% #1 - сам ответ
% #2 - кол-во баллов за ответ
% #3 - кол-во знаков после запятой
% #4 - допустимая погрешность
\newcommand{\answer}[4]%
{%
   \ifthenelse{\isundefined{\setplaceholders}}%
   {#1}%
   {
       \{\#\theanswercounter\}
   }%
  \stepcounter{answercounter}
}%
\begin{document}

\begin{pycode}
from random import randint
def rrstr(x, n): # округление до n знаков после запятой
    fmt = '{:.' + str(n) + 'f}'
    return fmt.format(x)#.replace('.',',')
N = randint(5, 17)
n = randint(3, N)
Omega = [randint(2, 7) for x in range(N)]
xo_mean = sum(Omega) / len(Omega)
e_sample_average = rrstr(xo_mean, 5)
sigma_2 = sum([(xi - xo_mean) ** 2 for xi in Omega]) / len(Omega)
var_sample_average = rrstr((sigma_2 / n) * ((N - n) / (N - 1)), 5)
itr = iter(list(range(1, N+1)))
x = 0
\end{pycode}

% name - название вопроса 
\begin{problem}[name="Случ. бесп. выборка"]
Признак $X(k)$ задан на множестве $\Omega$ = \{1, 2, . . . , $\py{N}$\} следующей таблицей:
\begin{center}
	\pys{\begin{tabular}{|c|!{f"{'c|'*N}"}}
	\hline
	$k$ & !{f"{' & '.join(map(str, list(range(1, N+1))))}"}\\
	\hline
	$X(k)$ & !{f"{' & '.join(map(str, Omega))} \\\\\n"}
	\hline
	\end{tabular}}
\end{center}

Из $\Omega$ извлекается случайная бесповторная выборка объема \py{n}. Найдите математическое ожидание и дисперсию среднего значения $\overline{X}$ признака $X$ в выборке.\\ 
Ответ введите с точностью до трех знаков после запятой.\\
\end{problem}

\begin{solution}
Математического ожидания $\mathbb E(\overline{X})$ = \answer{\py{e_sample_average}}{0.5}{3}{0.01}\\ 
Дисперсия $\mathbb Var(\overline{X})$ = \answer{\py{var_sample_average}}{0.5}{3}{0.01}
\end{solution}


\end{document}
