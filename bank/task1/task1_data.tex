\documentclass{article}
\usepackage{pgf,pgffor}% http://ctan.org/pkg/{pgf,pgffor}
\usepackage{xfp}
\usepackage{array}
\input{random}
\usepackage{lcg,calc}
\usepackage[russian]{babel}
\usepackage{amsfonts}
\usepackage{ifthen}
%\restartpythontexsession{\thesection}
%\usepackage{verbatim}   % for the comment environment
%%%%%%%%%%%%%%%%%%%%%%%%%%%%%%%%%%%
\newenvironment{problem}[1][]{\rmfamily}{}{%
  \small
}{%
}

\newcounter{answercounter}

% if you want to see the solution
%\newcommand{\hidesolution}{show them}
%\newcommand{\setplaceholders}{}
%%%%%%%%%%%%%%%%%%%%%%%%%%%%%%%%%%%%%%%
\newenvironment{solution}%
{%  
  \setcounter{answercounter}{1}
}%
{%
}
%%%%%%%%%%%%%%%%%%%%%%%%%%%%%%%%%
\setlength{\parindent}{0pt}
\usepackage[in]{fullpage}

% параметры:
% #1 - сам ответ
% #2 - кол-во баллов за ответ
% #3 - кол-во знаков после запятой
% #4 - допустимая погрешность
\newcommand{\answer}[4]%
{%
   \ifthenelse{\isundefined{\setplaceholders}}%
   {#1}%
   {
       \{\#\theanswercounter\}
   }%
  \stepcounter{answercounter}
}%
\begin{document}


% name - название вопроса 
\begin{problem}[name="Случ. бесп. выборка"]
Признак $X(k)$ задан на множестве $\Omega$ = \{1, 2, . . . , $6$\} следующей таблицей:
\begin{center}
	\begin{tabular}{|c|c|c|c|c|c|c|} 	\hline 	$k$ & 1 & 2 & 3 & 4 & 5 & 6\\ 	\hline 	$X(k)$ & 6 & 2 & 5 & 4 & 4 & 4 \\
 	\hline 	\end{tabular}
\end{center}

Из $\Omega$ извлекается случайная бесповторная выборка объема 4. Найдите математическое ожидание и дисперсию среднего значения $\overline{X}$ признака $X$ в выборке.\\ 
Ответ введите с точностью до трех знаков после запятой.\\
\end{problem}

\begin{solution}
Математического ожидания $\mathbb E(\overline{X})$ = \answer{4.16667}{0.5}{3}{0.01}\\ 
Дисперсия $\mathbb Var(\overline{X})$ = \answer{0.14722}{0.5}{3}{0.01}
\end{solution}


\end{document}
